% Chapter Template

\chapter{Background Information} % Main chapter title

\label{Chapter3} % Change X to a consecutive number; for referencing 
\section{What is Mixed Reality?}
Virtual Reality has been differentiated from Augmented Reality in 2014 by Tech Times. Virtual Reality is a virtual world that users can interact with. And Augmented Reality is users will continue to be involved with the real world while interacting with virtual objects around them. Nevertheless, as more and more companies claim that their products utilize either advantage of the Virtual Reality and Augmented Reality. Another technology has entered the picture: Mixed Reality. All these three technologies are developed to fool our brains and make our minds perceive objects as part of our reality that we are involved in. Compared with Virtual Reality (VR) and Augmented Reality (AR), Mixed Reality has the following special characteristics:
\\
1. VR has no direct interaction in the real world, the users are interacting with the pre-programmed digitized world which is delivered through the device. However, in Mixed Reality, the users are actually involved with the real world and interacting with the both real and digitalized object.
\\
2. In AR world, images are initialized to overlap with the user’s reality and merge with it. But it does not exactly care whether it fits in or not. It is simply there, but not affected by the users’ movement. But in Mixed Reality, these objects will respond to the change of users’ status. For example, when you walk far away from these images, and these images will get farer. They are able to change the angle by sitting, turning around or walking away. So, it will help the users to believe that these objects are real, and they exist around them.
\\
\\
That is why Mixed Reality is always presented as the technology which is aimed at taking the advantages of VR and AR, and immersing the user in a pre-programmed world while actually being anchored in the real world.

\section{Existing Mixed Reality Application}
The trend is more and more companies are focusing on Mixed Reality workspace. It will totally change the original working and studying behavior, improve the efficiency and help people to understand their work better. The following sections are some strong related researches about Mixed Reality Application which helps the design of Hybrid Desktop.
\subsection{Mixed Reality Desktop}
Based on some literature survey, a tangible mixed reality desktop has been implemented for digital media management in San Jose[18]. That system can support for gesture-oriented interactions in 3D space. Instead of attaching some devices to detect user’s hand or finger, that system uses computer vision techniques for hand detection, and location. But the limitation is it is a computer based program and it cannot provide the user an experience as immersed themselves in their real surroundings, and the gesture recognition is also limited. Inspired from this project, Hybrid Desktop is trying to discover a new way to interact with the real world in a more user friend way.
\\
\\
During the Leap Motion Hackathon 2015, a group was trying to augment a physical display with virtual widgets which strongly shows the prospect of interacting with real world in a brand-new way. They built an AR screen in the Oculus and use Leap Motion to interact with the windows. [3] This is labeled as the future of workspace and lots of people are eager to see what is going on. This new workshop frees people’s hands and turns the 2D workspace into 3D. All the browsers can be dragged from the screen and float in the air visually. Users are able to interact with these digital objects by their hands. 
\\
\\ 
“Augment” [4] is a company which is aimed at revolutionizing the way people preview, and edit. They are trying to display the designer’s 3D model to the real world by using augmented reality. Their new application called augment desktop provides a new experience and makes it easy and efficient to adjust model, gives people the freedom to configure materials for their work.
\\
\\
Except these applications, HoloLens[5] which is developed by Microsoft gives the Mixed Reality development a new chance. HoloLens is the first self-contained, holographic computer, enabling the user to engage with their digital content and interact with holograms in the world around them. It goes beyond the desktop and immerses into people’s life. The aim of this product is to bring ideas to life and transform the ways people communicate, create, collaborate and explore. Another similar headset Meta2[6] which is aimed at accessing to digital information and directing hand interaction with holograms. These two headsets have developed quickly last few years, made so many changes and overcame a lot of technical problems. But these two devices are sold in limited quantities and high prices. So not every university students are able to afford them.

\subsection{Mixed Reality Application in Education}
A search about Mixed reality in education, entertainment and training[19] indicates that the free learning education experience is the key to inspire curiosity, and it can help to create an optimistic attitude towards study. In the Mixed Reality space, the new world is equipped with a heightened sense of presence rather than just the projected image display, it surpasses the traditional visual projection system and presentation of information. 
\\
\\
In the Arts Center of Christchurch New Zealand, a research[20] related with turning an empty space into an Augmented Reality world for educational experience. They asked people to use see-through head mounted display to view 3D models of scientific data collaboratively, and ask them to analyze the data and work together. The results show that Augmented Reality is very suitable for collaborative tasks, and through this new experience, users can find the interface more intuitive and conducive to real world collaboration. It indicates that AR interfaces can offer seamless interaction between real and virtual world, and more researches should be involved in this field to explore how these characteristics can be applied in a school environment to enhance students studying experience.
\\
\\
Besides, desktop virtual reality has been considered as a powerful new technology for teaching and research in Industrial Education[21]. Because it enhanced the comprehension, increased student’s learning performance, and reduced training time. This fully immersive experiences offer a convincing illusion of participation and attract more people to changing their studying habits. However, there still presents both challenges and opportunities for researchers interested in this technology. The primary challenge is that there has been no conclusive research data to guide the design, and how can these applications relate with students learning process. More detailed research should be done to help this technology make more contribution in Education industrial.
\\
\\
In conclusion, Mixed Reality Application used in Education is significantly necessary for improving the existing study experience in university.



