% Chapter Template

\chapter{Further work} % Main chapter title

\label{Chapter8} % Change X to a consecutive number; for referencing 
\section{Improve experience with new device}
The limitation of the current project is the resolution ratio of the device and the comfort level of the headset. Because the current view is the camera view, so compared with seeing the real world, the quality of camera view is limited. Based on some research, Meta2 and HoloLens are two MR headsets which can solve this problem. When putting on them, the user is looking through the real world rather than the camera capture world. And it can only improve the resolution ratio but also expand the view. Also, the cardboard provides a lower user experience which will cause uncomfortable pressure around the head. The new device can also provide more comfortable wearing experience. 
\\
\\
Based on better wearing experience, users will be more willing to put themselves in Hybrid desktop for a long time and study in this environment. The implementation of Hybrid desktop can be moved to other platforms easily. Because Vuforia is compatible with HoloLens as well, it is convenient to rebuild the project and install it to a new device. HoloLens has its own gesture frame and voice input which will help the developer to implement more advanced functions.

\section{More Usability improvement}
The ultimate goal of the Hybrid desktop is to enable people work in this Mixed reality workstation, so all the laptop based application needs to be have a substitution in this Hybrid desktop. For example, there should be a web browser, Microsoft Word or Excel, Music player, etc. Anything can be done on laptop will be able to be done in the Hybrid desktop as well. 
\\
\\
When the web browser is accessible in the Hybrid desktop, the old bulletin board is not useful any more. It should be replaced by the Hybrid Desktop website. In other words, the bulletin board will be a web browser which display the Hybrid Desktop website as the default page. Besides, based on the user feedback, they hope they will be able to interact with the website more directly. So, replacing the bulletin board with the hybrid desktop website can help user to immerse themselves in this new world.
\\
\\
Another implementation this project is eager to explore more accurate location of hands. This project uses Leap Motion to sense the hands and respond with the gestures. However, the relative position of hands to Leap Motion is not the same as the relative position of hands people see. So, in the current project, the user is performing gestures to interact with the digital objects rather than grabbing it. In the future, this project hopes to let the user locate their hands more accurately and perform more advanced implementation, like grabbing objects and click objects.
\\
\\
Currently, there are limited number of PDFs in the application. But different student will have different PDF’s, and they all hope to access their own materials and manage it through a platform. This can be a further work based on the Hybrid Desktop website. People can upload their PDF on the website and then download it to the application. The application will convert the PDF to images automatically and then display it in the PDF reader.


\section{Introduce 3D printing to Hybrid Desktop}
People always dream about they can have their 3D model in real world. And it will be extremely useful for mechanical students. They can view their 3D model in the Hybrid desktop and adjust the parameters to design the best model, and when they finished their design, they can use 3D printing to get the real model directly. Then, they can compare the real one and the virtual one together. Besides, 3D printing can also be used for the function of copying and paste in Hybrid desktop. In the electronic world, people are only able to copy the electronic words, image, etc. But in Mixed reality world, except these digital objects, people will have the requirements to copy real objects. And 3D printing can help their dream to come true.
